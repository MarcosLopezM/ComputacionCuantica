\documentclass[./../main.tex]{subfiles}
\graphicspath{{img/}}

\begin{document}
    \section{}

    Si \(\op{H}\) es el operador Hadamard. Muestre que

    \begin{equation*}
        \op{H}{}{\otimes n}\ket{0}^{\otimes n} = \dfrac{1}{\sqrt{2^{2n}}}\sum_{i = 0}^{2^{n} - 1}\ket{i}.
    \end{equation*}

    \startsolution

    Reescribiendo \(\op{H}{}{\otimes n}\),

    \begin{equation*}
        \op{H}{}{\otimes n} = \bigotimes_{n} \op{H}\ket{0} = \bigotimes_{n} \ket{+} = \ket{+}^{\otimes n}.
    \end{equation*}

    Recordando que \(\ket{+} = \tfrac{1}{\sqrt{2}}(\ket{0} + \ket{1})\), tenemos que
    \begin{equation}
        \ket{+}^{\otimes n} = \dfrac{1}{\sqrt{2^{n}}}\sum_{i = 0}^{2^{n} - 1}\ket{i}.
        \label{eq:tensor_prod-plus_state}
    \end{equation}

    Si desarrollamos la expresión anterior para \(n = 2\),

    \begin{align*}
        \ket{+}^{\otimes 2} &= \dfrac{1}{\sqrt{2}}(\ket{0} + \ket{1}) \otimes \dfrac{1}{\sqrt{2}}(\ket{0} + \ket{1}),\\
        &= \dfrac{1}{2^{2 / 2}}(\ket{00} + \ket{01} + \ket{10} + \ket{11}).
    \end{align*}

    Donde cada estado corresponde a la representación binaria de los números \(0\) a \(3\). Entonces, \cref{eq:tensor_prod-plus_state} se puede generalizar a \(n\) estados como:

    \begin{equation*}
        \ket{+}^{\otimes n} = \dfrac{1}{2^{n / 2}}\sum_{i = 0}^{2^{n} - 1}\ket{i}.
    \end{equation*}

    Por lo tanto,

    \begin{empheq}[box=\resultbox]{equation*}
        \op{H}{}{\otimes n}\ket{0}^{\otimes n} = \dfrac{1}{\sqrt{2^{n}}}\sum_{i = 0}^{2^{n} - 1}\ket{i}.
    \end{empheq}
\end{document}
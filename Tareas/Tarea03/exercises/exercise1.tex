\documentclass[./../main.tex]{subfiles}
\graphicspath{{img/}}

\begin{document}
\problempts{25}
\section{}

Sean \(\op{U}\) una compuerta unitaria que actúa sore un qubit y \(\theta\in\bm{R}\) una constante. Muestre que:
\begin{enumerate}
	\item \(\op{U} \equiv \textrm{e}^{i\theta\op{U}}\), con \(i^{2} = -1\). \emph{Hint: considere la acción de cada compuerta sobre un qubit arbitrario, \(\ket{\psi}\)}.

	      \startsolution

	      Para determinar si ambos operadores son equivalentes, debemos analizar su acción sobre un qubit arbitrario \(\ket{\psi}\), tal que

	      \begin{align}
		      \op{U}\ket{\psi}                     & ,\label{eq:U_Psi}                                                    \\
		      \mathrm{e}^{i\theta}\op{U}\ket{\psi} & = \mathrm{e}^{i\theta}(\op{U}\ket{\psi}).\label{eq:Glob_Phase_U-Psi}
	      \end{align}

	      Es decir, el estado que resultado de \cref{eq:Glob_Phase_U-Psi} es el mismo que el de \cref{eq:U_Psi}, ya que \(\mathrm{e}^{i\theta}\) es una fase global, la cual no afecta las probabilidades o mediciones. Por lo tanto,

	      \begin{empheq}[box=\mainresult]{equation*}
		      \op{U} \equiv \mathrm{e}^{i\theta}\op{U}.
	      \end{empheq}

	      \pagebreak
	\item \(\cnot\op{U}\) no es una general equivalente a \(\cnot\textrm{e}^{i\theta}\op{U}\), donde \(\cnot\) denota una compuerta controlada. \emph{Hint: considere la acción de cada compuerta sobre un qubit arbitrario, \(\ket{\psi}\), usando como qubit de control a un qubit de la base computacional.}

	      \startsolution
	      Para probar que en general \(\cnot\op{U}\) no es equivalente a \(\cnot\mathrm{e}^{i\theta}\op{U}\), consideramos los diferentes casos de un qubit de control en la base computacional, \(\ket{0}\) y \(\ket{1}\), i.e.,
	      \begin{equation*}
		      \ket{0\psi} \qquad \ket{1\psi}.
	      \end{equation*}

	      \begin{enumerate}
		      \item \(\ket{0\psi}\)

		            Para \(\cnot\op{U}\),

		            \begin{equation*}
			            \cnot\op{U}\ket{0\psi} = \ket{0\psi}.
		            \end{equation*}

		            Y, para \(\cnot\mathrm{e}^{i\theta}\op{U}\),

		            \begin{equation*}
			            \cnot\mathrm{e}^{i\theta}\op{U}\ket{0\psi} = \ket{0\psi}.
		            \end{equation*}

		            Por lo tanto, cuando el qubit de control es \(\ket{0}\), ambas compuertas actúan de la misma manera, i.e.,

		            \begin{empheq}[box=\mainresult]{equation*}
			            \cnot\op{U} \equiv \cnot\mathrm{e}^{i\theta}\op{U}.
		            \end{empheq}

		      \item \(\ket{1\psi}\)

		            Para \(\cnot\op{U}\),

		            \begin{equation*}
			            \cnot\op{U}\ket{1\psi} = \tensprod{1}{\op{U}\ket{\psi}}.
		            \end{equation*}

		            Y,

		            \begin{equation*}
			            \cnot\mathrm{e}^{i\theta}\op{U}\ket{1\psi} = \tensprod{1}{\mathrm{e}^{i\theta}\op{U}\ket{\psi}}.
		            \end{equation*}

		            Por lo tanto, cuando el qubit de control es \(\ket{1}\),

		            \begin{empheq}[box=\mainresult]{equation*}
			            \cnot\op{U} \not\equiv \cnot\mathrm{e}^{i\theta}\op{U},
		            \end{empheq}

		            ya que la fase \(\mathrm{e}^{i\theta}\) se convierte en una fase relativa, pues tiene efecto en el qubit \(\ket{\psi}\) y no en el estado completo.
	      \end{enumerate}
\end{enumerate}
\end{document}

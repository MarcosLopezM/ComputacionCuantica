\documentclass[./../main.tex]{subfiles}
\graphicspath{{img/}}
\begin{document}
\problempts{10}

\section{}

Justifique por qué decimos que con una llamada a \(f\) podemos determinar con seguridad en qué caja está el regalo.

\startsolution

El por qué decimos que con una sola llamada a \(f\) podemos determinar en qué caja está el regalo se debe a que no estamos interesados en saber el valor de \(f(00)\), \(f(01)\), \(f(10)\) o \(f(11)\), sino que lo que nos interesa es saber el comportamiento de \(f\) en general. Es decir, con una sola llamada a \(f\) podemos determinar si \(f\) es constante o balanceada, y por lo tanto, podemos saber en qué caja está el regalo.
\end{document}

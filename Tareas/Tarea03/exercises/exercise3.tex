\documentclass[./../main.tex]{subfiles}
\graphicspath{{img/}}
\begin{document}
\problempts{50}

Se nos promete que una de cuatro cajas dadas contiene un regalo. Las cajas están enumeradas como \(00, 01, 10\) y \(11\). La función \(f \colon \{0,1\}^2 \to \{0,1\}\) devuelve \(1\) si la caja contiene el regalo y un \(0\) en caso contrario.
A continuación se muestra que podemos descubrir en qué caja se encuentra el regalo con sólo una llamada a esta función. Comenzamos inicializando el estado cuántico \(\ket{++-}\) y aplicando la función \(f\). Luego, transformamos los dos primeros qubits utilizando el siguiente circuito y finalmente realizamos la medición.

\begin{figure}[htb]
	\centering
	\includegraphics[width=0.6\textwidth]{circuito-algoritmo.pdf}
\end{figure}

\subsection*{Demostración}

\begin{itemize}
	\item \(\ket{\psi_{0}} = \ket{001}\)
	\item \(\ket{\psi_{1}} = \op{H}\ket{0}\op{H}\ket{0}\op{H}\ket{1} = \ket{++-} = \tfrac{1}{2\sqrt{2}}\sum_{x\in \{0,1\}^{2}}\)
	\item \(\ket{\psi_{2}} = \tfrac{1}{2\sqrt{2}}\sum_{x\in \{0,1\}^{2}}\ket{x}(\ket{f(x)} -\ket{1 \oplus f(x)}) = \tfrac{1}{2\sqrt{2}}\sum_{x\in \{0,1\}^{2}}(-1)^{f(x)}\ket{x}(\ket{0} - \ket{1})\)

	      Notemos que el tercer qubit regresa al estado \(\ket{-}\), así que podemos omitirlo en el resto del cálculo. Por esta razón, diremos que

	      \begin{equation*}
		      \ket{\psi_{2}} = \tfrac{1}{2}\sum_{x\in \{0,1\}^{2}}(-1)^{f(x)}\ket{x}.
	      \end{equation*}
	\item \(\ket{\psi_{3}} = (\tensprod{\op{I}}{\op{H}}\ket{\psi_{2}} = \tfrac{1}{2}\bigl((-1)^{f(00)}\ket{0+} + (-1)^{f(01)}\ket{0-} + (-1)^{f(10)}\ket{0+} + (-1)^{f(11)}\ket{1-}\biggr).\)
	\item \(\ket{\psi_{4}} = \cnot\ket{\psi_{3}} = \tfrac{1}{2}\bigl((-1)^{f(00)}\ket{0+} + (-1)^{f(01)}\ket{0-} + (-1)^{f(10)}\ket{0+} + (-1)^{f(11)}\ket{1-}\biggr).\)
	\item \(\ket{\psi_{5}} = (\tensprod{\op{H}}{\op{I}}\ket{\psi_{4}} = \tfrac{1}{2}\bigl((-1)^{f(00)}\ket{++} + (-1)^{f(01)}\ket{+-} + (-1)^{f(10)}\ket{-+} - (-1)^{f(11)}\ket{--}\biggr).\)
\end{itemize}

Consideremos todos los resultados posibles de la medición:

\begin{itemize}
	\item \textbf{Caso \(\bm{f(00) = 1}\):}. Si \(f(00) = 1\), entonces \(f(01) = f(10) = f(11) = 0\), por lo que el estado post-medición es

	      \begin{equation*}
		      \ket{\psi}^{\text{p.m.}} = \tfrac{1}{2}\bigl(-\ket{++} + \ket{+-} + \ket{-+} - \ket{--}\bigr) \propto \ket{11}.
	      \end{equation*}


	\item \textbf{Caso \(\bm{f(01) = 1}\):}. Si \(f(01) = 1\), entonces \(f(00) = f(10) = f(11) = 0\), por lo que el estado post-medición es

	      \begin{equation*}
		      \ket{\psi}^{\text{p.m.}} = \tfrac{1}{2}\bigl(\ket{++} - \ket{+-} + \ket{-+} - \ket{--}\bigr) \propto \ket{01}.
	      \end{equation*}

	\item \textbf{Caso \(\bm{f(10) = 1}\):}. Si \(f(10) = 1\), entonces \(f(00) = f(01) = f(11) = 0\), por lo que el estado post-medición es

	      \begin{equation*}
		      \ket{\psi}^{\text{p.m.}} = \tfrac{1}{2}\bigl(\ket{++} + \ket{+-} - \ket{-+} - \ket{--}\bigr) \propto \ket{10}.
	      \end{equation*}

	\item \textbf{Caso \(\bm{f(11) = 1}\):}. Si \(f(11) = 1\), entonces \(f(00) = f(01) = f(10) = 0\), por lo que el estado post-medición es

	      \begin{equation*}
		      \ket{\psi}^{\text{p.m.}} = \tfrac{1}{2}\bigl(\ket{++} + \ket{+-} + \ket{-+} + \ket{--}\bigr) \propto \ket{00}.
	      \end{equation*}
\end{itemize}

Por lo tanto, con una sola llamada a la función podemos encontrar el regalo.
\section{}

Repita los cálculos anteriores con todo detalle; es decir, justifique cada igualdad presentada. Es suficiente que muestre hasta el caso \(f(00  \Rightarrow \ket{\psi}^{\text{p.m.}} \propto \ket{11}\), el resto de los casos los puede dar por sentados.

\startsolution

Primero obtenemos el estado inicial del sistema:

\begin{align}
	\ket{\psi_{0}} & = \tensprod{\ket{0}}{\ket{0}}[\ket{1}][nopar],\nonumber \\
	\ket{\psi_{0}} & = \ket{001}.\label{eq:psi0}
\end{align}

Ahora, aplicamos las compuertas Hadamard, recordando que

\begin{align*}
	\op{H}\ket{0} & = \ket{+} = \dfrac{\ket{0} + \ket{1}}{\sqrt{2}}, \\
	\op{H}\ket{1} & = \ket{-} = \dfrac{\ket{0} - \ket{1}}{\sqrt{2}}.
\end{align*}

Así,

\begin{align*}
	\ket{\psi_{1}} & = (\tensprod{\op{H}}{\op{H}}[\op{H}][nopar])\ket{001},           \\
	               & = \tensprod{\op{H}\ket{0}}{\op{H}\ket{0}}[\op{H}\ket{1}][nopar], \\
	               & = \tensprod{\ket{+}}{\ket{+}}[\ket{-}][nopar],                   \\
	               & = \tensprod{\ket{+}^{\otimes 2}}{\ket{-}}.
\end{align*}

Por el resultado del problema 5 de la tarea 1, sabemos que

\begin{align}
	\ket{\psi_{1}}             & = \tfrac{1}{2}\sum_{i = 0}^{3}\tensprod{\ket{i}}{\ket{-}},\nonumber                                                 \\
	                           & = \tfrac{1}{2}\sum_{x\in \{0,1\}^{2}}\tensprod{\ket{x}}{\ket{-}},\nonumber                                          \\
	                           & = \tfrac{1}{2}\sum_{x\in \{0,1\}^{2}}\tensprod{\ket{x}}{\bigl(\dfrac{\ket{0} - \ket{1}}{\sqrt{2}}\Biggr)},\nonumber \\
	\Aboxedmain{\ket{\psi_{1}} & = \dfrac{1}{2\sqrt{2}}\sum_{x\in \{0,1\}^{2}}\ket{x}(\ket{0} - \ket{1})}.\label{eq:psi1}
\end{align}

Para obtener el estado \(\ket{\psi_{2}}\), primero recordamos la definición del oráculo,

\begin{equation*}
	\op{U}{f}\ket{x}\ket{y} = \ket{x}\ket{y \oplus f(x)},
\end{equation*}

entonces

\begin{align*}
	\ket{\psi_{2}} & = \dfrac{1}{2\sqrt{2}}\sum_{x\in \{0,1\}^{2}}\op{U}{f}\ket{x}(\ket{0} - \ket{1}),                                                   \\
	               & = \dfrac{1}{2\sqrt{2}}\sum_{x\in \{0,1\}^{2}}\ket{x}(\ket{f(x)} - \ket{1 \oplus f(x)}).\qquad \text{Ver pag. 7 Quantum Parallelism} \\
\end{align*}

Entonces, para cada \(x\), \(f(x)\) es \(0\) o \(1\), así que podemos escribir

\begin{enumerate}
	\item \(f(x) = 0\)

	      \begin{equation*}
		      \ket{f(x)} = \ket{0} \implies \ket{1 \oplus 0} = \ket{1}.
	      \end{equation*}

	      Entonces,

	      \begin{equation*}
		      = \dfrac{1}{2}\sum_{x\in \{0,1\}^{2}}\ket{x}\ket{-}.
	      \end{equation*}

	\item \(f(x) = 1\)

	      \begin{equation*}
		      \ket{f(x)} = \ket{1} \implies \ket{1 \oplus 1} = \ket{0}.
	      \end{equation*}

	      Entonces,

	      \begin{equation*}
		      = -\dfrac{1}{2}\sum_{x\in \{0,1\}^{2}}\ket{x}\ket{+}.
	      \end{equation*}
\end{enumerate}

Por lo tanto,

\begin{equation*}
	\ket{\psi_{2}} = \dfrac{1}{2}\sum_{x\in \{0,1\}^{2}}(-1)^{f(x)}\ket{x}\ket{-}.
\end{equation*}

Pero como el tercer qubit regresa al estado \(\ket{-}\), podemos omitirlo en el resto del cálculo. Por esta razón, diremos que

\begin{empheq}[box=\mainresult]{equation}
	\ket{\psi_{2}} = \dfrac{1}{2}\sum_{x\in \{0,1\}^{2}}(-1)^{f(x)}\ket{x}.\label{eq:psi2}
\end{empheq}

Antes de continuar, desarrollamos \cref{eq:psi2}:,

\begin{equation*}
	\ket{\psi_{2}} = \dfrac{1}{2}\Bigl((-1)^{f(00)}\ket{00} + (-1)^{f(01)}\ket{01} + (-1)^{f(10)}\ket{10} + (-1)^{f(11)}\ket{11}\Bigr).
\end{equation*}

Aplicamos la compuerta \(\op{H}\),

\begin{align*}
	\ket{\psi_{3}} & = (\tensprod{\op{I}}{\op{H}})\ket{\psi_{2}},                                                                         \\
	               & = \dfrac{1}{2}\Bigl((-1)^{f(00)}\ket{0+} + (-1)^{f(01)}\ket{0-} + (-1)^{f(10)}\ket{1+} + (-1)^{f(11)}\ket{1-}\Bigr).
\end{align*}

Para obtener el estado \(\ket{\psi_{4}}\), aplicamos la compuerta \(\cop\) a \(\ket{\psi_{3}}\):

\begin{align*}
	\ket{\psi_{4}} & = \cop\ket{\psi_{3}},                                                                                                                \\
	               & = \dfrac{1}{2}\Bigl[(-1)^{f(00)}\cop\ket{0+} + (-1)^{f(01)}\cop\ket{0-} + (-1)^{f(10)}\cop\ket{1+} + (-1)^{f(11)}\cop\ket{1-}\Bigr].
\end{align*}

Calculamos cada uno de los términos:

\begin{align*}
	\dfrac{\cop\ket{00}}{\sqrt{2}} + \dfrac{\cop\ket{01}}{\sqrt{2}} & = \dfrac{\ket{00} + \ket{01}}{\sqrt{2}} = \ket{0+},  \\
	\dfrac{\cop\ket{00}}{\sqrt{2}} - \dfrac{\cop\ket{01}}{\sqrt{2}} & = \dfrac{\ket{00} - \ket{01}}{\sqrt{2}} = \ket{0-},  \\
	\dfrac{\cop\ket{10}}{\sqrt{2}} + \dfrac{\cop\ket{11}}{\sqrt{2}} & = \dfrac{\ket{11} + \ket{10}}{\sqrt{2}} = \ket{1+},  \\
	\dfrac{\cop\ket{10}}{\sqrt{2}} - \dfrac{\cop\ket{11}}{\sqrt{2}} & = \dfrac{\ket{11} - \ket{10}}{\sqrt{2}} = -\ket{1-}.
\end{align*}

Entonces,

\begin{empheq}[box=\mainresult]{equation}
	\ket{\psi_{4}} = \dfrac{1}{2}\Bigl[(-1)^{f(00)}\ket{0+} + (-1)^{f(01)}\ket{0-} + (-1)^{f(10)}\ket{1+} - (-1)^{f(11)}\ket{1-}\Bigr].\label{eq:psi4}
\end{empheq}

Finalmente, para calcular el estado \(\ket{\psi_{5}}\), aplicamos la compuerta Hadamard a \cref{eq:psi4},

\begin{empheq}[box=\mainresult]{equation}
	\ket{\psi_{5}} = (\tensprod{\op{H}}{\op{I}})\ket{\psi_{4}} = \dfrac{1}{2}\Bigl[(-1)^{f(00)}\ket{++} + (-1)^{f(01)}\ket{+-} + (-1)^{f(10)}\ket{-+} - (-1)^{f(11)}\ket{--}\Bigr].\label{eq:psi5}
\end{empheq}

Ahora, consideraos el primer caso de los posibles resultados de la medición:

\begin{enumerate}
	\item \textbf{Caso \(\bm{f(00) = 1}\):}. Si \(f(00) = 1\), entonces \(f(01) = f(10) = f(11) = 0\), por lo que el estado post-medición es

	      \begin{align*}
		      \ket{\psi}^{\text{p.m.}}             & = \dfrac{1}{2}\Bigl((-1)^{1}\ket{++} + (-1)^{0}\ket{+-} + (-1)^{0}\ket{-+} - (-1)^{0}\ket{--}\Bigr),                                                                                                                                                                                                                                                                                                                                                                                            \\
		                                           & = \dfrac{1}{2}\Bigl(-\ket{++} + \ket{+-} + \ket{-+} - \ket{--}\Bigr),                                                                                                                                                                                                                                                                                                                                                                                                                           \\
		                                           & = \dfrac{1}{2}\bigl[-\Biggl(\tfrac{\ket{0} + \ket{1}}{\sqrt{2}}\Biggr)\Biggl(\tfrac{\ket{0} + \ket{1}}{\sqrt{2}}\Biggr) + \Biggl(\tfrac{\ket{0} + \ket{1}}{\sqrt{2}}\Biggr)\Biggl(\tfrac{\ket{0} - \ket{1}}{\sqrt{2}}\Biggr) + \Biggl(\tfrac{\ket{0} - \ket{1}}{\sqrt{2}}\Biggr)\Biggl(\tfrac{\ket{0} + \ket{1}}{\sqrt{2}}\Biggr) - \Biggl(\tfrac{\ket{0} - \ket{1}}{\sqrt{2}}\Biggr)\Biggl(\tfrac{\ket{0} - \ket{1}}{\sqrt{2}}\Biggr)\Biggr],                                                  \\
		                                           & = \begin{multlined}[t]\dfrac{1}{4}\bigl[-\ket{00} - \ket{01} - \ket{10} - \ket{11} + \ket{00} - \ket{01} + \ket{10} + \ket{11}\\ + \ket{00} + \ket{01} - \ket{10} - \ket{11} - \ket{00} + \ket{01} + \ket{10} - \ket{11}\Biggr],\end{multlined} \\
		                                           & = \dfrac{1}{4}\Bigl[-4\ket{11}\Bigr],                                                                                                                                                                                                                                                                                                                                                                                                                                                           \\
		                                           & = -\ket{11},                                                                                                                                                                                                                                                                                                                                                                                                                                                                                    \\
		      \Aboxedmain{\ket{\psi}^{\text{p.m.}} & \propto \ket{11}.}
	      \end{align*}
\end{enumerate}
\end{document}

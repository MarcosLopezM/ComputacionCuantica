\documentclass[./../main.tex]{subfiles}
\graphicspath{{img/}}
\begin{document}
\problempts{50}

Se nos promete que una de cuatro cajas dadas contiene un regalo. Las cajas están enumeradas como \(00, 01, 10\) y \(11\). La función \(f \colon \{0,1\}^2 \to \{0,1\}\) devuelve \(1\) si la caja contiene el regalo y un \(0\) en caso contario.
A continuación se muestra que podemos descubrir en qué caja se encuentra el regalo con sólo una llmada a esta función. Comenzamos inicializando el estado cuántico \(\ket{++-}\) y aplicando la función \(f\). Luego, transformamos los dos primeros qubits utilizando el siguiente circuito y finalmente realizamos la medición.

\begin{figure}[htb]
	\centering
	\includegraphics[width=0.6\textwidth]{example-image-a}
\end{figure}

\subsection*{Demostración}

\begin{itemize}
	\item \(\ket{\psi_{0}} = \ket{001}\)
	\item \(\ket{\psi_{1}} = \op{H}\ket{0}\op{H}\ket{0}\op{H}\ket{1} = \ket{++-} = \tfrac{1}{2\sqrt{2}}\sum_{x\in \{0,1\}^{2}}\)
	\item \(\ket{\psi_{2}} = \tfrac{1}{2\sqrt{2}}\sum_{x\in \{0,1\}^{2}}\ket{x}(\ket{f(x)} -\ket{1 \oplus f(x)}) = \tfrac{1}{2\sqrt{2}}\sum_{x\in \{0,1\}^{2}}(-1)^{f(x)}\ket{x}(\ket{0} - \ket{1})\)

	      Notemos que el tercer qubit regresa al estado \(\ket{-}\), así que podemos omitirlo en el resto del cálculo. Por esta razón, diremos que

	      \begin{equation*}
		      \ket{\psi_{2}} = \tfrac{1}{2}\sum_{x\in \{0,1\}^{2}}(-1)^{f(x)}\ket{x}.
	      \end{equation*}
	\item \(\ket{\psi_{3}} = (\tensprod{\op{I}}{\op{H}}\ket{\psi_{2}} = \tfrac{1}{2}\biggl((-1)^{f(00)}\ket{0+} + (-1)^{f(01)}\ket{0-} + (-1)^{f(10)}\ket{0+} + (-1)^{f(11)}\ket{1-}\biggr).\)
	\item \(\ket{\psi_{4}} = \cnot\ket{\psi_{3}} = \tfrac{1}{2}\biggl((-1)^{f(00)}\ket{0+} + (-1)^{f(01)}\ket{0-} + (-1)^{f(10)}\ket{0+} + (-1)^{f(11)}\ket{1-}\biggr).\)
	\item \(\ket{\psi_{5}} = (\tensprod{\op{H}}{\op{I}}\ket{\psi_{4}} = \tfrac{1}{2}\biggl((-1)^{f(00)}\ket{++} + (-1)^{f(01)}\ket{+-} + (-1)^{f(10)}\ket{-+} - (-1)^{f(11)}\ket{--}\biggr).\)
\end{itemize}

Consideremos todos los resultados posibles de la medición:

\begin{itemize}
	\item \textbf{Caso \(\bm{f(00) = 1}\):}. Si \(f(00) = 1\), entonces \(f(01) = f(10) = f(11) = 0\), por lo que el estado pors-medición es

	      \begin{equation*}
		      \ket{\psi}^{\text{p.m.}} = \tfrac{1}{2}\bigl(-\ket{++} + \ket{+-} + \ket{-+} - \ket{--}\bigr) \propto \ket{11}.
	      \end{equation*}


	\item \textbf{Caso \(\bm{f(01) = 1}\):}. Si \(f(01) = 1\), entonces \(f(00) = f(10) = f(11) = 0\), por lo que el estado post-medición es

	      \begin{equation*}
		      \ket{\psi}^{\text{p.m.}} = \tfrac{1}{2}\bigl(\ket{++} - \ket{+-} + \ket{-+} - \ket{--}\bigr) \propto \ket{01}.
	      \end{equation*}

	\item \textbf{Caso \(\bm{f(10) = 1}\):}. Si \(f(10) = 1\), entonces \(f(00) = f(01) = f(11) = 0\), por lo que el estado post-medición es

	      \begin{equation*}
		      \ket{\psi}^{\text{p.m.}} = \tfrac{1}{2}\bigl(\ket{++} + \ket{+-} - \ket{-+} - \ket{--}\bigr) \propto \ket{10}.
	      \end{equation*}

	\item \textbf{Caso \(\bm{f(11) = 1}\):}. Si \(f(11) = 1\), entonces \(f(00) = f(01) = f(10) = 0\), por lo que el estado post-medición es

	      \begin{equation*}
		      \ket{\psi}^{\text{p.m.}} = \tfrac{1}{2}\bigl(\ket{++} + \ket{+-} + \ket{-+} + \ket{--}\bigr) \propto \ket{00}.
	      \end{equation*}
\end{itemize}

Por lo tanto, con una sola llamada a la función podemos encontrar el regalo.
\section{}

Repita los cálculos anteriores con todo detalle; es decir, justifique cada igualdad presentada. Es suficiente que muestre hasta el caso \(f(00  \Rightarrow \ket{\psi}^{\text{p.m.}} \propto \ket{11}\), el resto de los casos los puede dar por sentados.
\end{document}
